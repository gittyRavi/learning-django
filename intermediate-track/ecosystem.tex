\documentclass[xcolor=dvipsnames]{beamer}

\usepackage{hyperref}
\usetheme{Warsaw}
\usecolortheme{beaver}
\setbeamertemplate{items}[circle]
\setbeamertemplate{blocks}[rounded=false,shadow=false]

\title{The Django Ecosystem}
\subtitle{and you}

\author{Trey Hunner}
\institute{Django Day}
\date{February 23, 2013}

\begin{document}

\begin{frame}[plain]
  \titlepage
\end{frame}

\begin{section}{How to get help}

  \begin{frame}{Where should I go to get help?}
    How I seek help on a Django-related problem:

    \begin{enumerate}
      \item Think of search terms that describe the problem
      \item Search Duck Duck Go and/or Google
      \item Search StackOverflow
      \item Search documentation
      \item Discover more search terms and start again at step 1
    \end{enumerate}
  \end{frame}

  \begin{frame}{Refreshing your memory}
    You have a function/class but don't know how to use it.

    \begin{enumerate}
      \item For Django built-ins try \href{https://github.com/jacobian/djangome\#readme}{http://django.me}
      \item Consult the documentation
      \item Read the code
      \pause
      \item \bf Read the code
    \end{enumerate}
  \end{frame}

  \begin{frame}{How to read the code}
    How should I read the code?

    \begin{description}
      \item[Good] Find the code on Github and read in Firefox
      \item[Better] Find the code on your computer and read in your IDE
      \item[Best] Use ``jump to definition'' to open code in your IDE
    \end{description}

    \pause
    Jumping to code definitions is supported in PyCharm, Wing IDE, Vim, and many other IDEs.
  \end{frame}

\end{section}

\begin{section}{Community}

  \begin{frame}{Stuff you should consume}
    \begin{itemize}
      \item Books \& Videos
      \item Newsletters \& Mailing Lists
      \item IRC
      \item Blogs
    \end{itemize}
  \end{frame}

  \begin{frame}{Books \& Videos}
    Books:

    \begin{itemize}
      \item \href{https://django.2scoops.org/}{Two Scoops of Django} (not free, but really really useful)
      \item The \href{http://www.djangobook.com/en/2.0/index.html}{Django Book} (useful if you've never read it)
    \end{itemize}

    Videos:

    \begin{itemize}
      \item \href{http://pyvideo.org/}{pyvideo.org} (big index of Python talks)
      \item \href{https://pycon-2012-notes.readthedocs.org/en/latest/index.html}{Marc Abramowitz's PyCon 2012 Notes}
    \end{itemize}

  \end{frame}


  \begin{frame}{Mailing Lists}
    You should subscribe to:

    \begin{itemize}
      \item \href{http://pycoders.com/}{PyCoder's Weekly} (you must subscribe to this)
      \item \href{https://groups.google.com/forum/?fromgroups\#!forum/django-users}{django-users list}
      \item \href{https://groups.google.com/forum/?fromgroups\#!forum/django-developers}{django-developers list} (for Django devs to chat publicly)
      \item the mailing list of your favorite Django projects
    \end{itemize}
  \end{frame}

  \begin{frame}{IRC Rooms}

    {\bf First}: Register a nickname on FreeNode.net.

    {\bf Next}: Join IRC channels:

    \begin{itemize}
      \item \#django for general chatting and help (read \href{https://code.djangoproject.com/wiki/IrcFAQ}{the FAQ})
      \item \#sandiegopython (there's always someone sitting in there)
      \item the IRC channels for your favorite projects.  Examples:
        \begin{itemize}
          \item \#django-compressor
          \item \#restframework
          \item \#haystack
          \item \#fabric
        \end{itemize}
    \end{itemize}
  \end{frame}


  \begin{frame}{Blogs to follow}
    \begin{itemize}
      \item \href{https://www.djangoproject.com/weblog/}{Django Weblog}: the official blog
      \item \href{http://www.planetdjango.org/}{Planet Django}: reposts from everywhere
    \end{itemize}
  \end{frame}

  \begin{frame}{More blogs to read}
    These don't update frequently, but you should {\bf read the archives}.

    \begin{itemize}
      \item \href{https://www.djangoproject.com/weblog/}{Hackers Gonna Hack}
      \item \href{http://procrastinatingdev.com/}{Procrastinating Dev}
      \item \href{http://toastdriven.com/}{Toast Driven}
      \item \href{http://pydanny.com/}{Inside the head of PyDanny}
      \item \href{http://harkablog.com/}{Hark! A Blog}
      \item \href{http://jacobian.org/writing/}{Jacob Kaplan-Moss' Blog}
      \item \href{http://blog.roseman.org.uk/}{Daniel Roseman's Blog}
      \item \href{http://ericholscher.com/blog/}{Surfing in Kansas}
    \end{itemize}

    I added just enough blogs to fit on this slide.  There might be more.
  \end{frame}

\end{section}

\begin{section}{Finding code you can use}

  \begin{frame}{Where do I go to find code?}
    \begin{itemize}
      \item \href{http://djangosnippets.org/}{DjangoSnippets.org}: small bits of code
      \item \href{https://www.djangopackages.com/}{DjangoPackages.com}: large directory of Django apps
      \item \href{http://stackoverflow.com/}{StackOverflow.com}: answers often contain\ldots
      \begin{itemize}
        \item useful code snippets
        \item links to useful Django apps
      \end{itemize}
    \item \href{https://github.com/}{GitHub}
    \item \href{https://bitbucket.org/}{Bitbucket}
    \end{itemize}
  \end{frame}

  \begin{frame}{I found some code, should I use it?}
    I found some code on Github.  Should I use it?

    \begin{description}
      \item[Yes] Why write new code when you can reuse code?
      \item[No] Someone else wrote it.  It could be buggy.
      \item[Maybe] You need to evaluate your options.
    \end{description}
  \end{frame}

\end{section}

\begin{section}{Evaluating code}

  \begin{frame}{How to evaluate code you found online}
    When evaluating new code ask\ldots

    \begin{enumerate}
      \item Is it fresh?
      \item Is it well-tested?
      \item Is it popular?
      \item Is it hackable?
    \end{enumerate}
  \end{frame}

  \begin{frame}{Is the code fresh?}
    \begin{itemize}
      \item Does this work in the newest version of Django?
      \item When was the last commit?
      \item When was the last mentioned of this package on blogs/StackOverflow?
    \end{itemize}

    \bigskip
    Most relevant for full projects, but could apply to code snippets.
  \end{frame}

  \begin{frame}{Is the code well-tested?}
    \begin{itemize}
      \item Are there tests?
      \item Are they quality tests?
      \item What does the code coverage look like?
    \end{itemize}

    \bigskip
    Most relevant for projects that don't have a very large community.
  \end{frame}

  \begin{frame}{Is the code popular?}
    \begin{itemize}
      \item How many stars or followers are there?
      \item How many forks, committers, closed issues and pull requests?
      \item Is it recommended on any blogs or Q \& A websites?
    \end{itemize}

    \bigskip
    Most relevant for large projects, especially when you wouldn't want to maintain your own fork of the code.
  \end{frame}

  \begin{frame}{Is this code hackable?}
    \begin{itemize}
      \item Do you understand how the code works?
      \item Would you be comfortable fixing bugs or adding features?
      \item Are the project maintainers accepting of contributions?
      \item Would you be willing to maintain your own fork of the code?
    \end{itemize}

    \bigskip
    Most relevant for projects that may not fit your needs exactly.
  \end{frame}

\end{section}

\begin{section}{Contributing code}

  \begin{frame}{Reasons you should write open source code}
    \begin{itemize}
      \item You'll eventually find a bug\ldots you might as well fix it
      \item You'll eventually discover a feature you want to add
      \item It will bring you wealth, fame, and happiness
    \end{itemize}
  \end{frame}

  \begin{frame}{Alternatives to open source}
    Some alternatives to contributing to open source projects:

    \begin{itemize}
      \item Monkey-patching the code to get the behavior you want
      \item Fixing/updating the code and maintaining your own fork
      \item Begging the community to fix your bugs and feature requests
    \end{itemize}
  \end{frame}

  \begin{frame}{How to contribute code}
    When contributing code to an open source project:

    \begin{enumerate}
      \item Tell the author what you're planning to do (optional)
      \item Write tests that should pass after you're done
      \item Write the code, asking the community for advice as needed
      \pause
      \item Write all the tests you forgot to write in step 2
    \end{enumerate}
  \end{frame}

  \begin{frame}{In Summary}

    \begin{enumerate}
      \item Watch a video from PyVideos every day
      \item Read Two Scoops of Django
      \item Write code and stuff
    \end{enumerate}
  \end{frame}

\end{section}

\end{document}
